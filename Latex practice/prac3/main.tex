\documentclass{article}
\usepackage{graphicx} 
\usepackage{xcolor}
\usepackage{multicol}
\usepackage{multirow}
\title{CSE 300 Online 1}
\author{2005079}
\date{July 6, 2019}

\begin{document}
\maketitle
\tableofcontents

\section{Introduction}
This is the first online practice on \LaTeX.\\ 
We will do some styling in section 2.
\section{Styling}
This is \textbf{\textcolor{red}{bold}}. This is \textit{\textcolor{blue}{italic}} \& so on.
\section*{Use of special characters}
You can use special characters by preceding with \\. Another package can be used named as verbatim. Example: \textasciitilde !\textdollar \#\textbackslash@* $\hat{}$
\section{List}
There are 3 kinds of list.
\begin{enumerate}
    \item[i] Un-ordered
    \item[ii] Ordered
    \item[iii] Description 
\end{enumerate}
\newpage
Let's see an example of lists:
\begin{description}
    \item[CSE 311] Data Communication I
    \begin{itemize}
        \item Fourier 
        \item[] ADC 
    \end{itemize}
    \item[CSE 305] Computer Architecture 
    \begin{enumerate}
        \item MIPS 
        \begin{itemize}
            \item[(a)] \textit{commands}
            \item[(b)] \textit{pipeline}
        \end{itemize}
        \item Shared memory
    \end{enumerate}
    \item[CSE 309] Compiler 
\end{description}

\section{Table}
For understanding the use of multi-row and multi-column command see the table 1
\begin{table}[h]
    \centering
    \begin{tabular}{|c|c|c|r|}
        \hline
        \multirow{2}{*}{Dollar} & Rupee & taka & Euro \\
        \cline{2-4}
        & \multicolumn{2}{c|}{\multirow{3}{*}{Dhaka}} & DU \\
        \cline{4-4}
        City & \multicolumn{2}{c}{} & \multicolumn{1}{r|}{BUET} \\
        \cline{1-1}
        \cline{4-4}
        Food & \multicolumn{2}{c|}{} & Mosque \\
        \hline
    \end{tabular}
    \caption{Caption}
    \label{tab:my_label}
\end{table}
\end{document}
